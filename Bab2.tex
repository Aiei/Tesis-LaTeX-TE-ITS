\chapter{KAJIAN PUSTAKA}

Kajian pustaka merupakan rangkuman singkat yang komprehensif tentang semua materi terkait yang terdapat di dalam berbagai referensi. Bagian ini dapat disajikan dalam tampilan diskusi atau debat antar pustaka. Selain itu juga dapat menjelaskan tentang teknik, peralatan atau teknologi yang akan dan/atau telah digunakan dalam penelitian yang akan/sedang dilaksanakan. Uraian yang ditulis diarahkan untuk menyusun kerangka pendekatan atau konsep yang diterapkan dalam penelitian. Materi yang disampaikan diusahakan dari referensi terbaru dan sumber asli, misalkan dari jurnal, seminar, buku, dan sebagainya.

\section{Kajian Penelitian Terkait}

Kajian penelitian terkait memuat tentang hasil penelitian pendahuluan yang dapat merupakan penelitian yang dilakukan oleh orang lain dan/atau penulis sendiri. Hal ini dilakukan untuk melihat sejauh mana penelitian terkait judul tesis sudah dilakukan atau dipublikasikan, dan urgensi dari penelitian tesis.

\section{Teori Dasar}

Dasar teori merupakan semua teori yang dipilih berdasarkan kajian pustaka yang melatarbelakangi permasalahan penelitian tesis yang dilakukan. Dasar teori juga akan digunakan sebagai pedoman untuk mengerjakan penelitian lebih lanjut. Bentuk dasar teori dapat berupa uraian kualitatif, model atau persamaan matematis. Pembahasan teori diutamakan yang terkait dan menunjang penelitian tesis saja.

Semua referensi yang digunakan atau dikutip harus dicantumkan dalam daftar pustaka. Pengutipan dapat dilakukan dengan dua cara, yaitu menggunakan gaya Harvard atau IEEE. Untuk gaya Harvard, nama belakang pengarang dan tahun penerbitan/ publikasi harus dicantumkan setelah kutipan di dalam tanda kurung kecil, misal (Siregar, 2006). Sedangkan untuk gaya IEEE, penulisan hanya menggunakan nomor publikasi dalam kurung siku [1]. Penjelasan tentang pengutipan secara lengkap dapat dibaca di panduan pengutipan gaya IEEE (IEEE style) yang terdapat di Teras Teknik Elektro ITS.

Semua gambar dan tabel harus jelas/tidak kabur/buram. Ukuran huruf pada gambar dan tabel harus dapat dibaca oleh mata normal dengan mudah. Gambar dan tabel diletakkan di tengah halaman (center alignment). Contoh gambar dapat dilihat pada Gambar \label{gambar-gaya-gesek-pendulum}. Penjelasan gambar ataupun tabel sebaiknya dikutip dalam kalimat sebelum/setelah gambar/tabel tersebut, contoh pengutipan dalam teks: nilai parameter sistem pendulum-kereta yang digunakan dalam simulasi dan implementasi terdapat dalam Tabel 2.1.

Nomor dan judul tabel ditulis di sisi kiri di atas tabel. Nomor tabel disesuaikan dengan letak tabel tersebut di dalam bab, misalkan: Tabel 2.1. Parameter Sistem Pendulum-Kereta [2]. Judul tabel ditulis dengan cara title case kecuali untuk kata sambung dan kata depan. Tabel dibuat dengan jarak spasi 1 (lihat  Tabel 2.1).

\gambar
    {gaya-gesek-pendulum.png}
    {Gaya Gesek pada Sistem Pendulum-Kereta}
    {gambar-gaya-gesek-pendulum}
    {0.6}

\section{Model Fuzzy Takagi-Sugeno}

Secara umum, sistem nonlinear dapat ditulis sebagai $x = f(x,u)$  dengan x merupakan variabel keadaan dan $u$ adalah input kontrol. Untuk membangun model fuzzy Takagi-Sugeno (T-S), model linear dari sistem nonlinear diperoleh melalui linearisasi sistem terhadap beberapa titik operasi, $x_i^*$. Model linear tersebut memiliki bentuk sebagai berikut [3]:

\begin{equation}
x(t) = A_ix(t)+B_iu(t), i=1,2,3,...,r
\end{equation}

\noindent dengan

\begin{equation}
A_i = \left.\frac{\partial f(x)}{\partial x}\right|x=x_i^* ; B_i = \left.\frac{\partial f(x,u)}{\partial u}\right|x=x_i^*
%F=\left.\frac{\partial f}{\partial x}\right|_{\hat x_{k-1}}
\end{equation}

\section{Observer Fuzzy}

Metode kontrol fuzzy dapat dibedakan berdasarkan cara desain dan bla bla bla…