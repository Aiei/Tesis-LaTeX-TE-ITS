\heading{JUDUL TESIS MENGGUNAKAN UKURAN FONT 14, CETAK TEBAL, \emph{CENTER ALIGNMENT}, JARAK SPASI 1, DAN MENGGUNAKAN HURUF KAPITAL}

\begin{tabular}{lcl}
Nama mahasiswa &:& Nama Nama Nama\\
NRP &:& 01234567891011\\
Pembimbing &:& 1. Dr. Nama Nama Nama, ST., MT.\\
& & 2. Nama Nama, ST., M.Sc., Ph.D.\\
\end{tabular}

\vspace{1ex}

\heading{ABSTRAK}

Abstrak adalah ringkasan yang singkat dan padat dari Tesis. Fungsi abstrak adalah membantu pembaca agar dengan cepat dapat memperoleh gambaran umum dari tulisan (ilmiah) tersebut. Dalam abstrak, tidak boleh ada kutipan hasil penelitian dari penulis lain.

Abstrak tesis berisi motivasi, perumusan masalah, tujuan, metode, hasil dan kesimpulan dari penelitian tesis yang telah dilakukan. Dalam kesimpulan, hindari penulisan yang menunjukkan keragu-raguan.

Setiap paragraf pada abstrak dimulai masuk 1 tab (1,5 cm) dari batas margin kiri dengan justify alignment. Jumlah kata maksimum adalah 350 kata. Kata kunci harus dituliskan di bagian bawah abstrak dengan jarak 3 spasi dari akhir abstrak, dengan jumlah kata minimal tiga dan maksimal lima. Kata kunci dipilih kata penting yang merupakan pokok yang spesifik dalam Tesis. Penulisannya diurutkan berdasarkan abjad pertama dari kata kunci tersebut.

\vspace{6ex}

\noindent Kata kunci: (jumlah kata minimal tiga dan maksimal lima)